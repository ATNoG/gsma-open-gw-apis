\section{Authentication and Fraud Prevention}

This group of APIs specializes in authenticating devices and preventing fraud.
Here we identified one API that was achievable with the available
infrastructure, (even though we had to complete it), and that was the One Time
Password SMS API.

\subsection{One Time Password SMS}

This API consists of two main steps, sending a One Time Password (OTP) via SMS
and verifying wether the result was correct or not, therefore we have two
endpoints, one for each action.

When calling the send OTP endpoint a couple things happen. First an
authentication ID is generated in order to be able to verify the code, since it
is required to use this authentication ID in the verification step to be able
to associate the code stored with the code received by the device. Then the
code itself is generated. At this point we store the code and replace the
placeholder in the message to be sent with the real code. The code is not
stored in plain text for security reasons, instead a hash of the code is
stored. The final step is to send the actual SMS with the message and the code,
this is done by calling an SMSC's endpoint.

At this stage in the process the code has been sent to the desired device and
all information has been stored. In order to complete the verification the
verification endpoint must be called with the code received on the device. The
endpoint must also be called with the authentication ID in order to be able to
associate the received code with an authentication attempt. From here two
possibilities arise, either the code is correct and the authentication succeeds
or the code is incorrect and after some number of failed attempts the code gets
removed and a new authentication cycle needs to start. The verification step
requires the code that is being verified to be hashed, and for security reasons
the comparison is done in constant time to prevent timing attacks.
