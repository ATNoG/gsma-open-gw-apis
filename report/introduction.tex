\chapter{Introdução}

\section{Contexto}

O desenvolvimento de redes 5G e para além do 5G introduziram capacidades
avançadas no acesso, reconfiguração e flexibilidade das redes. Isto atrai o
interesse de desenvolvedores em várias indústrias que pretendem oferecer
soluções integradas aos seus clientes que fazem uso destas capacidades dos
operadores de telecomunicações. No entanto, as interfaces oferecidas pelos
operadores atualmente e definidas pelo \emph{3rd Generation Partnership
	Project} (\emph{3GPP}) expõem muita da complexidade dos operadores. Esta complexidade
dificulta o desenvolvimento das soluções integradas e necessita de conhecimento
da área, limitando a sua adoção.

O \emph{GSMA Open Gateway} é uma iniciativa da indústria para resolver este
problema. Isto é feito por um conjunto de \emph{Application Programming
	Interfaces} (\emph{API}) definidos pelo projeto \emph{CAMARA} que expõe várias
interfaces simplificadas das redes dos operadores. Estas \emph{APIs} podem ser
oferecidas pelos operadores diretamente ou por agregadores como
\emph{Hyperscalers}.

Contudo, a maioria dos operadores não oferece estas interfaces, mas espera-se
que o façam num futuro próximo. Vistas as capacidades do \emph{Instituto de
	Telecomunicações} (IT) em Aveiro, com acesso a uma infraestrutura 5G
\footnote{\url{https://5gainer.eu/}}\cite{ieee:5GAIner}, existe interesse em
integrar as \emph{APIs} do \emph{GSMA Open Gateway} na sua oferta.

\section{Objetivos}

Os objetivos delineados para o projeto foram os seguintes:

\begin{enumerate}
	\item Implementação das seguintes \emph{APIs} do \emph{GSMA Open Gateway}
	      através de uma integração com o \emph{Core} comercial 5G do \emph{IT}.
	      \begin{itemize}
		      \item Família de \emph{APIs} para obter informação de
		            dispositivos (\emph{Device Information}).
		      \item Família de \emph{APIs} para obter informação sobre a
		            \emph{Quality-of-Service} (\emph{QoS}) da rede e também para
		            fornecer \emph{QoS-on-Demand} (\emph{QoD}).
	      \end{itemize}
	\item Implementação da \emph{API} do \emph{GSMA Open Gateway} para obtenção
	      e verificação de \emph{One Time Passwords} (\emph{OTP}) por \emph{SMS}
	      através de um \emph{SMS Centre} (\emph{SMSC}) integrado na rede 5G do
	      IT como parte do trabalho do nosso projeto.
	\item Implementação das \emph{APIs} do \emph{GSMA Open Gateway} para
	      serviços de localização através do projeto \emph{NEFSim}, um emulador
	      do \emph{Network Exposure Function} (\emph{NEF}, explicado em detalhe
	      na secção \ref{sec:related_work_3gpp}). A utilização deste emulador é
	      necessária, pois a rede do IT atualmente não têm este serviço na sua
	      rede.
	\item Integração do projeto com o
	      \emph{OpenSlice}\footnote{\url{https://osl.etsi.org/}}, um
	      \emph{Operation Support System} (\emph{OSS}), permitindo a modalidade
	      de \emph{CAMARA as a Service} (\emph{CAMARAaaS}).
	      %\item Integração do \emph{Common API Framework} (\emph{CAPIF}) com o
	      %      projeto, através do projeto \emph{OpenCAPIF}, para controlar a
	      %      exposição e consumo dos \emph{APIs}.
\end{enumerate}
