\chapter{Introduction}

\section{Context}

The development of 5G networks and beyond 5G have introduced
advanced capabilities in the access, reconfiguration and
flexibility of these networks. This attracts the interest of
software developers in several industries, aiming to offer
integrated solutions to their clients that leverage these
capabilities of the network and their operators. However, the
existing interfaces currently defined by the \emph{3rd Generation
Partnership Project} (\emph{3GPP}) expose much of the complexity
related to the networks and the operators. This complexity
hinders the development of integrated solutions and requires
extensive knowledge on the topic of networks and how to operate
them, limiting the adoption.

The \emph{GSMA Open Gateway} is an industry initiative to solve
this problem. This is done by a set of Application Programming
Interfaces (API) defined by the \emph{CAMARA} project, that
exposes various simplified interfaces of the operator networks.
These APIs can then be offered either through the MNOs directly
or by aggregators, such as \emph{Hyperscalers}.

However, most operatiors do not offer said interfaces, but it is
to be expected that they will do so in the near future. Given the
capabilities of \emph{Instituto de Telecomunicações} (IT) in
Aveiro, with access to a 5G infrastructure
\footnote{\url{https://5gainer.eu/}}\cite{ieee:5GAIner}, there is 
interest in integrating \emph{GSMA Open Gateway} APIs in their
offer.

\section{Goals}

The objectives outlined for the project were as follows:

\begin{enumerate} 
  \item Implemetation of the following \emph{GSMA
      Open Gateway} \emph{APIs} through integration with
      \emph{IT's} 5G Core. 
    \begin{itemize}
      \item \emph{API} family to obtain \emph{Device Information}
      \item \emph{API} family to obtain information on the
        network's \emph{Quality-of-Service} (\emph{QoS}) and to
        offer \emph{QoS-on-Demand} (\emph{QoD}).
    \end{itemize}

  \item Implementation of the \emph{GSMA
    Open Gateway} \emph{API} to obtain and verify \emph{One Time
    Passwords} (\emph{OTP}) through \emph{SMS}, utilizing an 
    \emph{SMS Centre} (\emph{SMSC}) integrated in \emph{IT's} 5G
    Network as a part of our project.

  \item Implementation of the \emph{GSMA Open Gateway APIs} to
    offer location through the \emph{NEFSim} project, an emulator
    of the \emph{Network Exposure Function} (\emph{NEF},
    explained in more detail in section
    \ref{sec:related_work_3gpp}) The utilization of this emulator
    is required, since \emph{IT's} network currently does not have
    such a service in its infrastructure.

  \item Integration of our project with the 
    \emph{OpenSlice}\footnote{\url{https://osl.etsi.org/}}, an
    \emph{Operation Support System} (\emph{OSS}), allowing for
    the usage of a \emph{CAMARA as a Service} (\emph{CAMARAaaS})
    business model.

  % \item Integration of the \emph{Common API Framework}
  %   (\emph{CAPIF}) with our project, through \emph{OpenCAPIF}, to
  %   control the exposure and consumption of our \emph{APIs}

\end{enumerate}
