
\section{Location Service APIs}

For the gateway we developed a set of Location APIs that we considered to be
most relevant and that were achievable with the available tools. These
consisted of the Location Retrieval, Location Verification and lastly
Geafencing Subscriptions APIs.

\subsection{Location Retrieval}

This API has a the goal of getting the location of a device with a simple API.
This location is as seen by network operators and will have varying degrees of
certainty.

The implementation uses 3GPP's Monitoring Event API with the monitoring type
set to \Verb{LOCATION_REPORTING}  with a single report. This means that instead of
sending a notification on the next location update the report is sent imediatly
on the response from where we extract the most recent information device
location information.

\subsection{Location Verification}

The Location Verification API, although it might seem similar to the previous
one has a different purpose. This API tells the caller wether a device is
within a certain area or not. This means that exact position of the device is
not relevant but only if it is or is not within a specified region.

Once again this API uses the Monitoring Event API on the NEF side to obtain the
location of the device. Having done this the gateway then calculates wether the
device is within or outside the specified region.

\subsection{Geofencing Subscription}

Finally in the Location Service collection of APIs we have the Geofencing
Subscriptions. This one is responsible for sending notifications to a given
sink. The notification is triggered everytime a device enters a given region or
whenever the device exits it depending on the configuration of the
subscription. Additionally, there are endpoints for managing the subscriptions,
allowing listing, getting details and deleting existing subscriptions.

As with the previous two APIs, this one also uses the Monitoring Event API
provided by the NEF to obtain the device location, but unlike the other ones it
actually creates a subscriptions. Internally the gateway has webhook listening
for location updates. Upon the creation of the subscription the details are
stored in a Redis instance for fast retrieval, each subscription will have a
unique identifier and, as such, when the Monitoring Event subsctipription is
created the webhook for the notification comes with that identifier. This way
when the gateway receives a notification from NEF it can distinguish between
the different subscriptions.

Once the notification is received on the gateway the location reported is
evaluated to know whether the device is or isn't inside the region and
depending on the previous known state and the configuration of the subscription
an according notification will be sent (if the conditions are met).

Along side the notification sending there are two more aspects that need
tracking which is the expire date of the subscription and the max number of
notifications. Every time a notification is sent, if a max number of
notifications is set, then a counter is incremented to stop notifications when
the limit is reached. As for the expire time, there is a seperate thread that
checks periodicly if the subscription has expired or not to flag it
accordingly.
