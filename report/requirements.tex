\chapter{Requisitos do Sistema}

Para o desenvolvimento do sistema foram recolhidos os diversos requisitos:

\section{Requisitos Funcionais}

\begin{itemize} \item \textbf{Serviços de Localização:} \begin{itemize}

		      \item O sistema deve permitir a criação de subscrição a enventos para que os
		            clientes da \emph{API CAMARA} recebam notificações ao entrar ou sair de uma
		            área definida \item O sistema deve permitir a obtenção da localização de um
		            dispositivo, sendo esta descrita por um círculo (coordenadas do centro e
		            raio) ou um polígono simples (lista de coordenadas) \item O sistema deve
		            permitir verificar se um dispositivo se encontra em uma certa área. Se esta
		            área se encontra fora da área coberta pelo operador ou se esta área for
		            válida, entre outros, deve retornar os erros corretos. \item O sistema deve
		            permitir obter a estimativa da densidade populacional numa área especificada
	      \end{itemize}

	\item \textbf{Qualidade de comunicação:} \begin{itemize}

		      \item O sistema deve permitir a troca de informação relevante para auxiliar
		            na tomada de decisões relacionadas às APIs da rede. \item O sistema deve
		            permitir que desenvolvedores consultem a rede sobre a probabilidade de
		            atender aos requisitos de conectividade numa sessão. \item O sistema deve
		            permitir que desenvolvedores possam definir diferentes QoS, nos AP dos
		            clientes, dependendo da sua necessidade de largura de banda. \item O
		            sistema deve permitir a obtenção dos perfis de QoS existentes numa rede,
		            retornando assim a informação relativa aos mesmos. \item O sistema deve
		            permitir a atribuição de perfis QoS a um dispositivo. A rede aplicará o
		            perfil de QoS ao tráfego do dispositivo sempre que estiver conectado, até
		            que este aprovisionamento seja removido. \end{itemize}

	\item \textbf{Proteção contra fraudes:} \begin{itemize} \item O sistema
		            deve permitir a verificação de um número de telefone, através do envio e
		            validação de uma OTP. \end{itemize}

	\item \textbf{Obtenção de informação sobre dispositivos:} \begin{itemize}

		      \item O sistema deve permitir verificar se um determinado dispositivo está
		            disponível na rede. \item O sistema deve permitir que um subscritor da
		            API seja notificado caso exista uma alteração na disponibilidade de um
		            determinado dispositivo. \item O sistema deve permitir verificar se um
		            determinado dispositivo se encontra numa situação de roaming, caso isto
		            seja verdade, retorna a informação existente sobre o país. \item O
		            sistema deve permitir que um subscritor da API seja notificado caso
		            exista alguma alteração no estado de roaming de um dispositivo.
	      \end{itemize} \end{itemize}

\section{Requisitos Não Funcionais}

\begin{itemize} \item \textbf{Desempenho e Latência:} \begin{itemize} \item O
		            sistema deve manter tempos de resposta consistentes à medida que a base de
		            utilizadores ou o volume de dados aumenta \item O sistema deve conseguir
		            tratar um número crescente de pedidos. \end{itemize}

	\item \textbf{Escalabilidade:} \begin{itemize} \item O sistema deve
		            garantir que o sistema suporte um elevado volume de pedidos e
		            dispositivos conectados \item O sistema deve escalar horizontalmente
		            e/ou verticalmente consoante a procura. \end{itemize}

	\item \textbf{Disponibilidade e Fiabilidade:} \begin{itemize} \item O
		            sistema deve estabelecer níveis de disponibilidade (por exemplo, SLA
		            de 99,9\% de uptime) \item O sistema deve apresentar um nível de
		            fiabilidade elevado (isto é, 99.9999\% de fiabilidade) \item O
		            sistema deve ter um mecanismo de redundância e tolerância a falhas,
		            garantindo a operação contínua mesmo em condições adversas. \end{itemize}
	\item  \textbf{Interoperabilidade:} \begin{itemize} \item O sistema deve ter
		            compatibilidade com diversas operadoras, garantindo a integração das APIs
		            com a infraestrutura de diferentes operadoras de rede, promovendo a
		            universalidade dos serviços. \item O sistema deve assegurar que as APIs
		            sigam padrões abertos e melhores práticas, promovendo a interoperabilidade e a
		            aceitação universal entre os diversos sistemas e tecnologias. \end{itemize}


	\item \textbf{Manutenibilidade e Extensibilidade:} \begin{itemize} \item O
		            sistema deve ter uma arquitetura modular, permitindo dividir o sistema em
		            componentes independentes, facilitando atualizações e correções sem afetar
		            o funcionamento global. \item O sistema deve ter uma documentação
		            detalhada, com manuais, de modo a ajudar os desenvolvedores a compreender e
		            manter o sistema eficazmente. \item O sistema deve ter código limpo e
		            legível, seguindo padrões que tornam o código fácil de entender e
		            modificar, reduzindo a probabilidade de erros. \item O sistema deve ter
		            flexibilidade para atualizações, facilitando a implementação de novas
		            funcionalidades ou melhorias sem a necessidade de reescrever grandes partes do
		            código. \end{itemize} \item \textbf{Usabilidade:} \begin{itemize} \item O
		            sistema deve retornar, caso existam, erros seguindo o padrão definido,
		            facilitando a integração destas APIs. \item O sistema deve retornar somente a
		            informação relevante ao desenvolvedor, abstraindo-o das informações mais
		            complexas da rede \end{itemize} \end{itemize}
