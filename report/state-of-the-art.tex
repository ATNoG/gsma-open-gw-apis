\chapter{Estado da Arte}

\section{Projetos Relacionados}

Esta secção visa apresentar outros projetos na mesma área e/ou com objetivos
semelhantes ao nosso projeto, a relação destes para com o nosso projeto, e como
estes se comparam com o nosso projeto.

\subsection{3GPP OpenAPIs}\label{sec:related_work_3gpp}

O \emph{3GPP}, como corpo responsável pelas normas que definem o 5G, delineou
não só a interface rádio usada para o 5G (denominada \emph{5G New Radio} e
encurtada para \emph{5G-NR}) mas também a arquitetura do \emph{core} da rede
dos operadores de serviço. Esta nova arquitetura é baseada em serviços
(\emph{Service-Based Architecture}) denominados \emph{Network Functions}
(\emph{NFs}), e consistem em aplicações que correm em hardware genérico, como o
que é oferecido por fornecedores de infraestrutura \emph{cloud}.

Estes serviços comunicam todos por \emph{APIs} definidos pelo \emph{3GPP} e
descritas em \emph{OpenAPIs}, uma norma para documentação de \emph{APIs}. No
entanto, estas \emph{APIs} são desenhadas para uso apenas dentro da rede
\emph{core} do operador e não para uso por terceiros, com a exceção das
\emph{APIs} oferecidas pelo \emph{Network Exposure Function} (\emph{NEF}).

O \emph{NEF} é uma \emph{NF} especificamente desenhada para permitir que
terceiros acedam as capacidades da rede do operador de maneira segura. As
\emph{APIs} oferecidas pelo \emph{NEF} são designadas por \emph{Northbound
	Interface} e os seus clientes de \emph{Application Functions} (\emph{AFs}).


Concluindo, as \emph{3GPP OpenAPIs} operam somente no domínio do operador e não
expõe as capacidades da rede do operador a terceiros, exceto o \emph{NEF}, que
tem o objetivo explícito de suportar a integração de terceiros com a rede do
operador. Mesmo que um operador de rede decida expor diretamente as \emph{APIs}
definidas pelo \emph{3GPP}, os desenvolvedores de software não seriam capazes de
as utilizar, uma vez que estas se limitam ao escopo da rede, pelo que para as utilizar
seria necessário bastante conhecimento específico ao domínio da rede.

\subsection{CAMARA}

O \emph{CAMARA} é uma iniciativa que tem como principal objetivo a definição,
desenvolvimento e validação de \emph{APIs} de \emph{Network-as-a-Service}
(\emph{NaaS})\cite{10.1145/3538401.3546825}. Estas \emph{APIs} estão a ser
desenvolvidas com um foco na sua implementação rápida para alcançar um dos
grandes objetivos do \emph{CAMARA}, isto é, a sua adoção pelos operadores de
modo a garantir que haja uma interface unificada, que não dependa dos seus
fornecedores, e que consiga dar a terceiros a oportunidade de conseguir
interagir com os serviços 5G sem a necessidade de contactar os operadores.
Outro grande objetivo do \emph{CAMARA} é a acessibilidade das suas \emph{APIs}
para desenvolvedores sem experiência na área do 5G e das telecomunicações, isto
é, garantir que os desenvolvedores consigam utilizar os serviços de forma
simples e sem a necessidade de interagir diretamente com os fornecedores ou
aprender detalhes de telecomunicações para conseguir utilizar as \emph{APIs} do
\emph{3GPP}.

Esta iniciativa está fortemente relacionada com o \emph{GSMA Open Gateway} que iremos referir
de seguida, uma vez que as \emph{APIs} serão publicadas através do projeto
\emph{CAMARA}\cite{5gamericas}. Além disto, o \emph{GSMA Open Gateway}, o
\emph{CAMARA} e o \emph{TMF} estão alinhados para conseguir trazer ao mercado
as capacidades do 5G de um modo mais acessível e estandardizado.

\subsection{GSMA Open Gateway}

De modo semelhante ao \emph{NEF}, o \emph{GSMA Open Gateway} tem por objetivo
expor as diversas capacidades dos operadores de rede a terceiros.
Contudo, existem vantagens em utilizar o \emph{GSMA Open Gateway}:

\begin{itemize}
	\item \textbf{Simplificação}, o \emph{NEF} expõe interfaces bastante
	      complexas com vários detalhes da rede do operador, requerendo
	      conhecimento no domínio de redes móveis e 5G, aumentando o custo da
	      adoção destas \emph{APIs}.
	\item \textbf{Agregação}, o \emph{NEF} expõe apenas acesso à rede do
	      operador, enquanto o \emph{GSMA Open Gateway} prevê a existência de
	      agregadores que oferecem as \emph{APIs} para vários operadores de uma
	      maneira unificada, requerendo contacto apenas com um agregador invés de
	      múltiplos operadores.
	\item \textbf{Gestão de privacidade}, o \emph{GSMA Open Gateway}
	      contempla mecanismos para a gestão de consentimento dos utilizadores
	      finais que em muitos casos é necessário por motivos legais.
\end{itemize}

Isto não implica que o \emph{NEF} é uma interface inferior, este oferece uma
maior integração com a rede dos operadores. Além disso, é possível utilizar o
\emph{NEF} para implementar o \emph{GSMA Open Gateway}, modelando este como uma
\emph{AF}. Na realidade, o \emph{NEF} e o \emph{GSMA Open Gateway} são entre si
tecnologias complementares.

Ainda assim, é importante realçar que o \emph{GSMA Open Gateway} é uma
interface mais cómoda à maioria dos desenvolvedores de aplicações de terceiros,
pois, ao contrário da interação direta com o \emph{NEF}, esta não requer
conhecimento do domínio de redes móveis.

O \emph{NEF} acaba por ter também a sua importância, por um lado por oferecer
uma maior integração com a rede do operador, mas por outro por ser possível de
implementar sobre este um \emph{GSMA Open Gateway}.

\subsection{TMF Open Digital Architecture}

O \emph{Open Digital Architecture} (\emph{ODA}) é uma \emph{framework} criada
pelo \emph{TM Forum} (\emph{TMF}) para ajudar fornecedores de serviços a
transacionarem para soluções abertas e interoperáveis. Esta \emph{framework} é
transversal a várias áreas tocando não só na parte técnica mas também nos
processos da empresa.

No nosso projeto o mais relevante são os \emph{OpenAPIs} definidos nesta
\emph{framework} que definem interfaces independentes do domínio para oferta de
recursos e serviços. De particular interesse são as interfaces que permitem a
obtenção de um catálogo de serviços, a aquisição destes serviços e o controlo sob
o seu ciclo de vida.

As \emph{APIs TMF} são mais genéricas que as oferecidas pelo \emph{GSMA Open
	Gateway}, oferecendo mais operações, mas também aumentando a complexidade do
seu uso em relação às oferecidas pelo \emph{GSMA Open Gateway}. O \emph{GSMA
	Open Gateway} tem também a vantagem de acomodar o cenário de agregação de
vários \emph{Gateways}. Este permite que um agregador ofereça acesso a
\emph{APIs} do \emph{GSMA Open Gateway} de vários operadores de maneira
unificada, sem que o cliente tenha que manter contacto com cada operador
individualmente.

Um caso interessante na colaboração destes dois projetos é uso das \emph{APIs
	TMF} para permitir a oferta do \emph{GSMA Open Gateway} como um serviço pelos
operadores e agregadores. Neste cenário, as \emph{APIs TMF} seriam utilizadas
para oferecer ao cliente a oportunidade de adquirir um serviço que
disponibilizaria um \emph{GSMA Open Gateway}. Este seria orquestrado e 
disponibilizado de maneira automatizada, reforçando a oferta do operador/agregador e
simplificando o processo do desenvolvedor em obter acesso ao \emph{GSMA Open
	Gateway}.



\section{Adoção das \emph{APIs}}

Apesar de estas iniciativas serem relativamente novas, já existem operadores a
implementar as \emph{APIs} definidas pelo \emph{CAMARA} e pelo \emph{GSMA Open
	Gateway}.

A \emph{Ericsson}, enquanto empresa multinacional de telecomunicações, começou
a implementar as \emph{APIs} definidas pelo projeto \emph{CAMARA}
\footnote{\url{https://www.fierce-network.com/wireless/ericsson-leads-top-operators-new-global-api-venture}}.
Em conjunto com outras empresas como \emph{Google Cloud}, \emph{Deutsche
	Telekom}, \emph{Reliance Jio} e \emph{T-Mobile}
\footnote{\url{https://adunaglobal.com/\#mission}}, a \emph{Ericsson} participa
então agora de uma iniciativa denominada de \emph{Aduna}, estando estes a
desenvolver e adotar um conjunto de \emph{APIs} \emph{CAMARA}. As primeiras
\emph{APIs} a serem suportadas serão focadas maioritariamente em prevenção de
fraude, tendo por objetivo a verificação de números atribuídos a clientes e uma
\emph{API} que visa facilitar o processo de \emph{SIM Swapping}
\footnote{\url{https://www.fierce-network.com/wireless/what-will-aduna-ericssons-massive-api-venture-release-first}}.

De igual modo, na região do Médio Oriente e Norte de África, também uma
operadora, a \emph{Ooredoo}, implementou recentemente as \emph{APIs} de
\emph{CAMARA}, de modo a conseguir oferecer novas oportunidades a negócios
(clientes).
\footnote{\url{https://www.ooredoo.com/en/media/news_view/ooredoo-group-leads-mena-with-gsma-camara-api-implementation/}}

Isto reafirma o compromisso que as diversas operadoras fizeram para com o
\emph{GSMA Open Gateway} e com a iniciativa \emph{CAMARA}.

Tal como \emph{Aduna}, também a \emph{Ooredoo} começou por oferecer \emph{SIM
	Swapping} programático, oferecendo também o necessário para a implementação de
\emph{One Time Password} (\emph{OTP}) via \emph{SMS}.

Por fim, os quatro principais operadores móveis de França (\emph{Bouygues
	Telecom}, \emph{Free}, \emph{Orange} e \emph{SFR}) anunciaram uma parceria para
lançar \emph{APIs} de rede que ajudam a combater fraudes online e proteger
identidades digitais. Como parte da iniciativa global \emph{GSMA Open Gateway},
estas disponibilizarão três \emph{APIs} no mercado francês, sendo estas
\emph{Know Your Customer} (\emph{KYC}) \emph{Match}, \emph{SIM Swapping} e
\emph{Number Verification}. Estas foram testadas com bancos e empresas na
França e devem ser lançadas comercialmente no 1.º semestre de 2025.
\footnote{\url{https://www.gsma.com/newsroom/press-release/french-mobile-industry-accelerates-deployment-of-network-apis-through-gsma-open-gateway-initiative/}}
