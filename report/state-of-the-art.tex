\chapter{State of the Art}

\section{Related Work}

This section aims to present projects in the same field of work,
or at least with similar objectives to our own, the relationship
between these other projects and ours, and how they compare.

\subsection{3GPP OpenAPIs}\label{sec:related_work_3gpp}

\emph{3GPP}, the body responsible for the norms that define the
5G protocol, designed not only the radio interface to be used by
5G Networks (designated \emph{5G New Radio}, shortened as
\emph{5G-NR}), but also the architecture of the network's
\emph{Core}. This new architecture is a \emph{Service-Based
Architecture}, with each service denominated as \emph{Network
Functions} (\emph{NFs}). These consist of applications that run
on generic hardware, such as the one offered by Cloud
Infrastructure.

These services communicate between them by \emph{APIs} defined by
\emph{3GPP} and described in \emph{OpenAPIs}, a documentation
standard. However, these \emph{APIs} are designed for use only
inside the MNO's network \emph{core}, and not to be used by third
parties, exception being made to the \emph{Network Exposure
Function} (\emph{NEF}). The latter, is an \emph{NF} specifically
designed to allow third-party access to the MNO's network
capabilities securely. The \emph{APIs} that are offered are
designated as \emph{Northbound Interfaces}, and their clients are
denominated of \emph{Application Functions} (\emph{AFs}).

In conclusion, 3GPP OpenAPIs operate only in the operator's
domain and do not expose the operator's network capabilities to
third parties, exception being made to the \emph{NEF}, whose
purpose is to explicitly support network integration by
third-parties with MNO's. Even if a MNO decides to directly
expose the \emph{3GPP} designed APIs, software developers would
not be capable of using them, since these are limited to the
scope of the network, therefore using them requires extensive
domain-specific knowledge.

\subsection{CAMARA}

The \emph{CAMARA} initiative's main objectives are the
definition, development and validation of
\emph{Network-as-a-Service} (\emph{NaaS}) \emph{APIs}. These are
being developed with focus on rapid implementation, in order to
achieve one of \emph{CAMARA}'s big objectives, which is MNO
adoption, as a way to ensure an unified interface that is not
supplier dependent and that is able to give third-parties the
opportunity to interact with 5G services without the need to
contact the MNO.

Another of \emph{CAMARA}'s big objectives is the accessibility of
all its APIs to software developers without any experience in the
5G/Telecom domain, that is, to ensure that the developers are
able to utilize these services simply and without the need to
interact directly with network operators or learn the details of
telecommunications to be able to use these \emph{APIs} 

This initiative is heavily related with \emph{GSMA Open Gateway},
which we will talk about in the next section, since it is these
\emph{APIs} that will be published through \emph{CAMARA}. In
addition, the GSMA Open Gateway, CAMARA and TMF are aligned to
bring 5G capabilities to the market in a more accessible and
standardized way.


\subsection{GSMA Open Gateway}

Similarly to the \emph{NEF}, the \emph{GSMA Open Gateway} aims to
expose the various capabilities of network operators to third
parties. However, there are advantages to using the \emph{GSMA
Open Gateway}:

\begin{itemize} \item \textbf{Simplification}, the NEF exposes
      quite complex interfaces with various details of the
      operator's  of the operator's network, requiring knowledge
      of mobile networks and 5G, increasing the cost of adopting
      these APIs.

    \item \textbf{Aggregation}, the NEF only exposes access to
      the operator's network, while the GSMA Open Gateway
      provides for the existence of aggregators that offer the
      APIs to various operators in an unified manner, requiring
      contact with only one aggregator rather than multiple
      operators.

\item \textbf{Privacy management}, the GSMA Open Gateway includes
mechanisms for the end-user consent, which in many cases is
necessary for legal reasons.\end{itemize}

This does not imply that \emph{NEF} is an inferior interface; it
offers greater integration with operators' networks. In reality,
\emph{NEF} and \emph{GSMA Open Gateway} are complementary
technologies. Still, it's important to note that \emph{GSMA Open
Gateway} is a more convenient interface for most third-party
application developers because, unlike direct interaction with
\emph{NEF},  it does not require knowledge of mobile networks.
The \emph{NEF} is also important, on the one hand because it
offers greater integration with the operator's network, but also
because it is possible to implement a \emph{GSMA Open Gateway}
over it, modelling it as an \emph{AF}.

\subsection{TMF Open Digital Architecture}

The \emph{Open Digital Architecture} (\emph{ODA}) is a framework
created by the \emph{TM Forum} (\emph{TMF}) to help service
providers transition to open and interoperable solutions. This
framework spans several areas, touching not only on the technical
side but also on the company's processes. In our project, the
most relevant are the \emph{OpenAPIs} defined in this framework,
which define domain-independent interfaces for offering resources
and services. Of particular interest are the interfaces that make
it possible to obtain a service catalog, acquire these services
and control their life cycle.

The \emph{TMF APIs} are more generic than those offered by the
\emph{GSMA Open Gateway}, offering more operations, but also
increasing the complexity of their use compared to those offered
by the \emph{GSMA Open Gateway}. The \emph{GSMA Open Gateway}
also has the advantage of accommodating the multiple gateway
aggregation scenario, allowing an aggregator to offer access to
\emph{GSMA Open Gateway APIs} from several operators in an unified
way, without the customer having to maintain contact with each
operator individually. An interesting case in the collaboration
of these two projects is the use of \emph{TMF APIs} to allow
operators and aggregators to offer \emph{GSMA Open Gateway} as a
service. In this scenario, the \emph{TMF APIs} would be used to
offer the customer the opportunity to purchase a service that
would provide a \emph{GSMA Open Gateway}. This would be
orchestrated and made available in an automated way, reinforcing
the operator/aggregator's offer and simplifying the developer's
process of gaining access to the \emph{GSMA Open Gateway}.


\section{\emph{API} Adoption} Although these initiatives are
relatively new, there are already operators implementing the
\emph{APIs} defined by \emph{CAMARA} and the \emph{GSMA Open
Gateway}.

\emph{Ericsson}, a multinational telecommunications company, has
started to implement the \emph{APIs} defined by the \emph{CAMARA}\footnote{\url{https://www.fierce-network.com/wireless/ericsson-leads-top-operators-new-global-api-venture}} project, along with companies such as \emph{Google Cloud},
\emph{Deutsche Telekom}, \emph{Reliance Jio} and
\emph{T-Mobile}\footnote{\url{https://adunaglobal.com/\#mission}},
participating in an initiative called \emph{Aduna}, which is
developing and adopting a set of CAMARA APIs. The first APIs to
be supported will be focused mainly on fraud prevention, with the
aim of verifying numbers assigned to customers and an API that
aims to facilitate the SIM Swapping
process\footnote{\url{https://www.fierce-network.com/wireless/what-will-aduna-ericssons-massive-api-venture-release-first}}.

Similarly, in the Middle East and North Africa region, an
operator, \emph{Ooredoo}, recently implemented \emph{CAMARA
APIs}, allowing the offer of new opportunities to businesses
(customers)\footnote{\url{https://www.ooredoo.com/en/media/news_view/ooredoo-group-leads-mena-with-gsma-camara-api-implementation/}}.

This reaffirms the commitment that the various operators have
made to the \emph{GSMA Open Gateway} and the \emph{CAMARA}
initiative.

Like \emph{Aduna}, \emph{Ooredoo} has also started to offer
programmatic \emph{SIM Swapping}, as well as providing the
necessary support for implementing One Time Password (OTP) via
SMS. Finally, the four main mobile operators in France
(\emph{Bouygues Telecom}, \emph{Free}, \emph{Orange} and
\emph{SFR}) announced a partnership to launch network \emph{APIs}
to help fight online fraud and protect digital identities. As
part of the \emph{GSMA Open Gateway} global initiative, they will
make three APIs available on the French market, namely \emph{Know
Your Customer (KYC) Match}, \emph{SIM Swapping} and \emph{Number
Verification}. These have been tested with banks and companies in
France and should be launched commercially in the first half of
2025\footnote{\url{https://www.gsma.com/newsroom/press-release/french-mobile-industry-accelerates-deployment-of-network-apis-through-gsma-open-gateway-initiative/}}.
