\chapter{Estado da Arte}

\section{Projetos Relacionados}

Esta secção visa apresentar outros projetos na mesma área e/ou com objetivos
semelhantes ao nosso projeto, a relação destes para com o nosso projeto, e como
estes se comparam com o nosso projeto.

\subsection{3GPP OpenAPIs}\label{sec:related_work_3gpp}

O \emph{3GPP} como corpo responsável pelas normas que definem o 5G, delineou
não só a interface rádio usada para o 5G (denominada \emph{5G New Radio} e
encurtada para \emph{5G-NR}) mas também a arquitetura do \emph{core} da rede
dos operadores de serviço. Esta nova arquitetura é baseada em serviços
(\emph{Service-Based Architecture}) denominados \emph{Network Functions}
(\emph{NFs}), e consistem em aplicações que correm em hardware genérico, como o
que é oferecido por fornecedores de infraestrutura \emph{cloud}.

Estes serviços comunicam todos por \emph{APIs} definidos pelo \emph{3GPP} e
descritas em \emph{OpenAPIs}, uma norma para documentação de \emph{APIs}. No
entanto, estas \emph{APIs} são desenhadas para uso apenas dentro da rede
\emph{core} do operador e não para uso por terceiros, com a exceção das
\emph{APIs} oferecidas pelo \emph{Network Exposure Function} (\emph{NEF}).

O \emph{NEF} é uma \emph{NF} especificamente desenhada para permitir que
terceiros acedam as capacidades da rede do operador de maneira segura. As
\emph{APIs} oferecidos pelo \emph{NEF} são designadas por \emph{Northbound
	Interface} e os seus clientes de \emph{Application Functions} (\emph{AFs}).


Concluindo, as \emph{3GPP OpenAPIs} operam somente no domínio do operador e não
expõe as capacidades da rede do operador a terceiros, exceto o \emph{NEF} que
têm o objetivo explícito de suportar a integração de terceiros com a rede do
operador. 

\subsection{GSMA Open Gateway}

De modo semelhante ao \emph{NEF}, o \emph{GSMA Open Gateway} tem por objetivo
expor as diversas capacidades dos operadores de rede a terceiros. 
Contudo, existem vantagens em utilizar o \emph{GSMA Open Gateway}:

\begin{itemize}
	\item \textbf{Simplificação}, o \emph{NEF} expõe interfaces bastante
	      complexas com vários detalhes da rede do operador, requerendo
	      conhecimento no domínio de redes móveis e 5G, aumentando o custo da
	      adoção destas \emph{APIs}.
	\item \textbf{Agregação}, o \emph{NEF} expõe apenas acesso à rede do
	      operador enquanto o \emph{GSMA Open Gateway} prevê a existência de
	      agregadores que oferecem as \emph{APIs} para vários operadores de uma
	      maneira unificada. Requerendo contacto apenas com um agregador invés de
	      múltiplos operadores.
	\item \textbf{Gestão de privacidade}, o \emph{GSMA Open Gateway}
	      contempla mecanismos para a gestão de consentimento dos utilizadores
	      finais que em muitos casos é necessário por motivos legais.
\end{itemize}

Isto não implica que o \emph{NEF} é uma interface inferior, este oferece uma
maior integração com a rede dos operadores. Além disso, é possível utilizar o
\emph{NEF} para implementar o \emph{GSMA Open Gateway}, modelando este como uma
\emph{AF}. Na realidade, o \emph{NEF} e o \emph{GSMA Open Gateway} são entre si tecnologias
complementares. 

Ainda assim é importante realçar que o \emph{GSMA Open Gateway} é uma interface mais cómoda à maioria
dos desenvolvedores de aplicações de terceiros, pois ao contrário da interação
diretamente com o \emph{NEF}, esta não requer conhecimento do
domínio de redes móveis. 

O \emph{NEF}, acaba por ter também a sua importância, por um lado por oferecer uma maior
integração com a rede do operador, mas por outro por ser capaz até de um \emph{GSMA Open Gateway} ser
implementado sobre o este.

\subsection{TMF Open Digital Architecture}

O \emph{Open Digital Architecture} (\emph{ODA}) é uma \emph{framework} criada
pelo \emph{TM Forum} (\emph{TMF}) para ajudar fornecedores de serviços a
transacionarem para soluções abertas e interoperáveis. Esta \emph{framework} é
transversal a várias áreas tocando não só na parte técnica mas também nos
processos da empresa.

No nosso projeto o mais relevante são os \emph{OpenAPIs} definidos nesta
\emph{framework} que definem interfaces independentes do domínio para oferta de
recursos e serviços. De particular interesse são as interfaces que permitem a
obtenção de um catálogo de serviços, a compra destes serviços, e o controlo sob
o seu ciclo de vida.

As \emph{APIs TMF} são mais genéricas que as oferecidas pelo \emph{GSMA Open
	Gateway}, oferecendo mais operações, mas também aumentando a complexidade do
seu uso em relação as oferecidas pelo \emph{GSMA Open Gateway}. O \emph{GSMA
	Open Gateway} tem também a vantagem de acomodar o cenário de agregação de
vários \emph{Gateways}. Este permite que um agregador oferece acesso a
\emph{APIs} do \emph{GSMA Open Gateway} de vários operadores de maneira
unificada, sem que o cliente tenha que manter contacto com cada operador
individualmente.

Um caso interessante na colaboração destes dois projetos é uso das \emph{APIs
	TMF} para permitir a oferta do \emph{GSMA Open Gateway} como um serviço pelos
operadores e agregadores. Neste cenário, as \emph{APIs TMF} seriam utilizadas
para oferecer ao cliente a oportunidade de comprar um serviço que
disponibilizaria um \emph{GSMA Open Gateway}. Este seria orquestrado e feito
disponível de maneira automatizada, reforçando a oferta do operador/agregador e
simplificando o processo do desenvolvedor em obter acesso ao \emph{GSMA Open
	Gateway}.

\subsection{CAMARA}

%% Falar sobre oq e NaaS

O \emph{CAMARA} é uma iniciativa que tem como principal objetivo a definição,
desenvolvimento e validação de \emph{APIs} de \emph{Network-as-a-Service}
(\emph{NaaS})\cite{10.1145/3538401.3546825}. Estas \emph{APIs} estão a ser
desenvolvidas com um foco na sua implementação rápida para alcançar um dos
grandes objetivos do \emph{CAMARA}, isto é, a sua adoção pelos operadores de
modo a garantir que haja uma interface unificada, que não dependa dos seus
fornecedores, e que consiga dar aos third parties a oportunidade de conseguir
interagir com os serviços 5G sem a necessidade de contactar os operadores.
Outro grande objetivo do \emph{CAMARA} é a acessibilidade das suas \emph{APIs}
para desenvolvedores sem experiência na área do 5G e das telecomunicações, isto
é, garantir que os desenvolvedores consigam utilizar os serviços de forma
simples e sem a necessidade de interagir diretamente com os fornecedores ou
aprender detalhes de telecomunicações para conseguir utilizar as \emph{APIs} do
\emph{3GPP}.

Esta iniciativa está fortemente relacionada com o \emph{GSMA Open Gateway}, uma
vez que as \emph{APIs} serão publicadas através do projeto
\emph{CAMARA}\cite{5gamericas}. Além disto, o \emph{GSMA Open Gateway}, o
\emph{CAMARA}, e o \emph{TMF} estão alinhados para conseguir trazer ao mercado
as capacidades do 5G de um modo mais acessível e estandardizado.
