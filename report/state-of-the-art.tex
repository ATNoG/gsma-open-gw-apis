\chapter{State of Art}

\section{Related Work}

This section aims to present projects in the same field of work,
or at least with similar objectives to our own, the relationship
between these other projects and ours, and how they compare.

\subsection{3GPP OpenAPIs}\label{sec:related_work_3gpp}

\emph{3GPP}, the body responsible for the norms that define the
5G protocol, designed not only the radio interface to be used by
5G Networks (designated \emph{5G New Radio}, shortened as
\emph{5G-NR}), but also the architecture of the network's
\emph{Core}. This new architecture is a \emph{Service-Based
Architecture}, with each service denominated as \emph{Network
Functions} (\emph{NFs}). These consist of applications that run
on generic hardware, such as the one offered by Cloud
Infrastructure.

These services communicate between them by \emph{APIs} defined by
\emph{3GPP} and described in \emph{OpenAPIs}, a documentation
standard. However, these \emph{APIs} are designed for use only
inside the MNO's network \emph{core}, and not to be used by third
parties, exception being made to the \emph{Network Exposure
Function} (\emph{NEF}). The latter, is an \emph{NF} specifically
designed to allow thrid-party access to the MNO's network
capabilities securely. The \emph{APIs} that are
offered are designated as \emph{Northbound Interfaces}, and their
clients are denominated of \emph{Application Functions}
(\emph{AFs}).

In conclusion, 3GPP OpenAPIs operate only in the operator's
domain and do not expose the operator's network capabilities to
third parties, exception being made to the \emph{NEF}, whose
purpose is to explicitly support network integration by
third-parties with MNO's. Even if a MNO decides to directly
expose the \emph{3GPP} designed APIs, software developers would
not be capable of using them, since these are limited to the
scope of the network, therefore using them requires extensive
domain-specific knowledge.

\subsection{CAMARA}

The \emph{CAMARA} initiative's main objectives are the
definition, development and validation of
\emph{Network-as-a-Service} (\emph{NaaS}) \emph{APIs}. These are
being developed with focus on rapid implementation, in order to
achieve one of \emph{CAMARA}'s big objectives, which is MNO
adoption, as a way to ensure a unified interface that is not
supplier dependant and that is able to give third-parties the
opportunity to interact with 5G services without the need to
contact the MNO.

Another of \emph{CAMARA}'s big objectives is the accessability of
all its APIs to software developers without any experience in the
5G/Telecom domain, that is, to ensure that the developers are able to
utilize these services simply and without the need to interact
directly with network operators or learn the details of
telecommunications to be able to use these \emph{APIs} 

This initiative is heavily related with \emph{GSMA Open Gateway},
which we will talk about in the next section, since it is these
\emph{APIs} that will be published through \emph{CAMARA}.
In addition, the GSMA Open Gateway, CAMARA and TMF are aligned
to bring 5G capabilities to the market in a more accessible and standardized way.


\subsection{GSMA Open Gateway}

De modo semelhante ao \emph{NEF}, o \emph{GSMA Open Gateway} tem por objetivo
expor as diversas capacidades dos operadores de rede a terceiros.
Contudo, existem vantagens em utilizar o \emph{GSMA Open Gateway}:

\begin{itemize}
	\item \textbf{Simplificação}, o \emph{NEF} expõe interfaces bastante
	      complexas com vários detalhes da rede do operador, requerendo
	      conhecimento no domínio de redes móveis e 5G, aumentando o custo da
	      adoção destas \emph{APIs}.
	\item \textbf{Agregação}, o \emph{NEF} expõe apenas acesso à rede do
	      operador, enquanto o \emph{GSMA Open Gateway} prevê a existência de
	      agregadores que oferecem as \emph{APIs} para vários operadores de uma
	      maneira unificada, requerendo contacto apenas com um agregador invés de
	      múltiplos operadores.
	\item \textbf{Gestão de privacidade}, o \emph{GSMA Open Gateway}
	      contempla mecanismos para a gestão de consentimento dos utilizadores
	      finais que em muitos casos é necessário por motivos legais.
\end{itemize}

Isto não implica que o \emph{NEF} é uma interface inferior, este oferece uma
maior integração com a rede dos operadores. Além disso, é possível utilizar o
\emph{NEF} para implementar o \emph{GSMA Open Gateway}, modelando este como uma
\emph{AF}. Na realidade, o \emph{NEF} e o \emph{GSMA Open Gateway} são entre si
tecnologias complementares.

Ainda assim, é importante realçar que o \emph{GSMA Open Gateway} é uma
interface mais cómoda à maioria dos desenvolvedores de aplicações de terceiros,
pois, ao contrário da interação direta com o \emph{NEF}, esta não requer
conhecimento do domínio de redes móveis.

O \emph{NEF} acaba por ter também a sua importância, por um lado por oferecer
uma maior integração com a rede do operador, mas por outro por ser possível de
implementar sobre este um \emph{GSMA Open Gateway}.

\subsection{TMF Open Digital Architecture}

O \emph{Open Digital Architecture} (\emph{ODA}) é uma \emph{framework} criada
pelo \emph{TM Forum} (\emph{TMF}) para ajudar fornecedores de serviços a
transacionarem para soluções abertas e interoperáveis. Esta \emph{framework} é
transversal a várias áreas tocando não só na parte técnica mas também nos
processos da empresa.

No nosso projeto o mais relevante são os \emph{OpenAPIs} definidos nesta
\emph{framework} que definem interfaces independentes do domínio para oferta de
recursos e serviços. De particular interesse são as interfaces que permitem a
obtenção de um catálogo de serviços, a aquisição destes serviços e o controlo sob
o seu ciclo de vida.

As \emph{APIs TMF} são mais genéricas que as oferecidas pelo \emph{GSMA Open
	Gateway}, oferecendo mais operações, mas também aumentando a complexidade do
seu uso em relação às oferecidas pelo \emph{GSMA Open Gateway}. O \emph{GSMA
	Open Gateway} tem também a vantagem de acomodar o cenário de agregação de
vários \emph{Gateways}. Este permite que um agregador ofereça acesso a
\emph{APIs} do \emph{GSMA Open Gateway} de vários operadores de maneira
unificada, sem que o cliente tenha que manter contacto com cada operador
individualmente.

Um caso interessante na colaboração destes dois projetos é uso das \emph{APIs
	TMF} para permitir a oferta do \emph{GSMA Open Gateway} como um serviço pelos
operadores e agregadores. Neste cenário, as \emph{APIs TMF} seriam utilizadas
para oferecer ao cliente a oportunidade de adquirir um serviço que
disponibilizaria um \emph{GSMA Open Gateway}. Este seria orquestrado e 
disponibilizado de maneira automatizada, reforçando a oferta do operador/agregador e
simplificando o processo do desenvolvedor em obter acesso ao \emph{GSMA Open
	Gateway}.



\section{Adoção das \emph{APIs}}

Apesar de estas iniciativas serem relativamente novas, já existem operadores a
implementar as \emph{APIs} definidas pelo \emph{CAMARA} e pelo \emph{GSMA Open
	Gateway}.

A \emph{Ericsson}, enquanto empresa multinacional de telecomunicações, começou
a implementar as \emph{APIs} definidas pelo projeto \emph{CAMARA}
\footnote{\url{https://www.fierce-network.com/wireless/ericsson-leads-top-operators-new-global-api-venture}}.
Em conjunto com outras empresas como \emph{Google Cloud}, \emph{Deutsche
	Telekom}, \emph{Reliance Jio} e \emph{T-Mobile}
\footnote{\url{https://adunaglobal.com/\#mission}}, a \emph{Ericsson} participa
então agora de uma iniciativa denominada de \emph{Aduna}, estando estes a
desenvolver e adotar um conjunto de \emph{APIs} \emph{CAMARA}. As primeiras
\emph{APIs} a serem suportadas serão focadas maioritariamente em prevenção de
fraude, tendo por objetivo a verificação de números atribuídos a clientes e uma
\emph{API} que visa facilitar o processo de \emph{SIM Swapping}
\footnote{\url{https://www.fierce-network.com/wireless/what-will-aduna-ericssons-massive-api-venture-release-first}}.

De igual modo, na região do Médio Oriente e Norte de África, também uma
operadora, a \emph{Ooredoo}, implementou recentemente as \emph{APIs} de
\emph{CAMARA}, de modo a conseguir oferecer novas oportunidades a negócios
(clientes).
\footnote{\url{https://www.ooredoo.com/en/media/news_view/ooredoo-group-leads-mena-with-gsma-camara-api-implementation/}}

Isto reafirma o compromisso que as diversas operadoras fizeram para com o
\emph{GSMA Open Gateway} e com a iniciativa \emph{CAMARA}.

Tal como \emph{Aduna}, também a \emph{Ooredoo} começou por oferecer \emph{SIM
	Swapping} programático, oferecendo também o necessário para a implementação de
\emph{One Time Password} (\emph{OTP}) via \emph{SMS}.

Por fim, os quatro principais operadores móveis de França (\emph{Bouygues
	Telecom}, \emph{Free}, \emph{Orange} e \emph{SFR}) anunciaram uma parceria para
lançar \emph{APIs} de rede que ajudam a combater fraudes online e proteger
identidades digitais. Como parte da iniciativa global \emph{GSMA Open Gateway},
estas disponibilizarão três \emph{APIs} no mercado francês, sendo estas
\emph{Know Your Customer} (\emph{KYC}) \emph{Match}, \emph{SIM Swapping} e
\emph{Number Verification}. Estas foram testadas com bancos e empresas na
França e devem ser lançadas comercialmente no 1.º semestre de 2025.
\footnote{\url{https://www.gsma.com/newsroom/press-release/french-mobile-industry-accelerates-deployment-of-network-apis-through-gsma-open-gateway-initiative/}}
